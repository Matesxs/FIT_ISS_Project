\section{Určení rušivých frekvencí}
\label{sec:noise_freq}

Z grafů výše můžeme vidět, že peaky rušení jsou +- stejně daleko od sebe, takže můžeme říci že jsou harmonicky vztažené neboli, že frekvence rušení 2., 3., 4., jsou násobky 1.
Z FFT grafu z úlohy 3 můžeme odečíst hodnoty frekvencí peaků, které jsou 3223, 2423, 1615 a 800 Hz.
Dále byla naprogramována funkce na získání frekvencí rušení z výkonové spektrální hustoty.
Funkce vrátila hodnoty 806.114, 1612.228, 2418.341, 3223.323 Hz.
Tyto hodnoty se velmi blíží těm co byli odečteny z grafu, takže funkce funguje správně.