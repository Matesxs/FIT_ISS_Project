\section{Úvod}

"Casto se stane, ze dostaneme signal zaruseny nejakymi artefakty a je potreba jej vyfiltrovat. Pro tento projekt 
jsme pro Vas pripravili signaly ze zname databaze TIMIT, kam se ale “zatoulaly” ctyri harmonicky vztazene
cosinusovky. Vasim ukolem je signal analyzovat, najit, na kterych frekvencich cosinusovky jsou, navrhnout filtr
nebo filtry pro cisteni signalu a pak signal vycistit."

V tomto projektu se budeme zabývat zkoumáním možností analýzy signálu a jeho filtrací podle zadání.
Dále prozkoumáme možnosti filtrování pomocí neuronových sítí a úskalí této metody.

Tento projekt je zpracován v jazyce python za použití knihoven numpy, scipy a matplotlib. Pro dobrovolnou část projektu kde zkoumáme možnosti filtrace pomocí neuronových sítí je použita knihovna tensorflow.

Projekt je složen z dvou nezávyslých částí. První částí je vypracování zadání a druhou částí je můj výzkum možností filtrace pomocí neuronových sítí.
Řešení části ze zadání lze najít v souboru \textbf{main.ipynb}, soubory \textbf{helpers.py}, \textbf{settings.py}, \textbf{process\textunderscore ai\textunderscore data.py}, \textbf{train\textunderscore ai.py} a \textbf{interface\textunderscore ai.py} patří k testování neuronových sítí.

Soubor \textbf{settings.py} obsahuje nastavení tvorby dat pro neuronovou síť, cesty k datům, parametry trénování, atd.
\textbf{helpers.py} obsahuje pomocné funkce pro zpracování dat a tvorbu dat, pomocné callbacky pro trénování, tvorbu modelu a čtení a předzpracování dat.
\textbf{process\textunderscore ai\textunderscore data.py} zpracovává data a vytváří datasety.
\textbf{train\textunderscore ai.py} vytváří model a trénuje ho.
\textbf{interface\textunderscore ai.py} slouží k interakci s modelem po jeho vytrénování.