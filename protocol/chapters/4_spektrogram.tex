\section{Spektrogram}

V této části se podíváme na frekvenční složení vstupního signálu. \\
Byl zobrazen spektrogram celého signálu pomocí funkce spektrogram z knihovny scipy a také jeho výkonová spektrální hustota. Místo této funkce by šla použít přímo naše fourierova transformace u předchozího kroku použítá na každý frame, ale bylo by zbytečné ji použít když máme k dispozici přímo funkci, která nám vygeneruje rovnou data celého spektrogramu bez nutnosti dalších operací s daty na naší straně. Další výhodou bude rychlost zpracování, protože tato knihovna je velmi optimalizovaná narozdíl od naší implementace. \\
Pro další krok - určení rušivých frekvencí, byl vytvořen i graf výkonové spektrální hustoty pro první frame signálu (pouze šum).

\begin{figure}[H] 
	\centering
	\includegraphics[scale=0.7,keepaspectratio]{Figure_6}
	\caption{Výkonový spektrogram}
\end{figure}

\begin{figure}[H] 
	\centering
	\includegraphics[scale=0.7,keepaspectratio]{Figure_7}
	\caption{Výkonová spektrální hustota}
\end{figure}

\begin{figure}[H] 
	\centering
	\includegraphics[scale=0.7,keepaspectratio]{Figure_26}
	\caption{Výkonová spektrální hustota frame \#0}
\end{figure}